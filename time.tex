\documentclass[10pt,oneside%,draft%
]{memoir}

% --- Packages ----------------------------------------------

\usepackage[USenglish]{babel}
\usepackage[utf8]{inputenc}
\usepackage[T1]{fontenc}
\usepackage{textcomp}
\usepackage{color}
\usepackage{graphicx}
\usepackage{IEEEtrantools}
\usepackage{verbatim}
\usepackage{tocloft}
\usepackage{amsmath}
\usepackage{amsfonts}
\usepackage{amssymb}
\usepackage{braket}
\usepackage[hyphens]{url}
\usepackage{makeidx}
\usepackage[colorlinks=true,urlcolor=blue,linkcolor=blue,linktocpage=true]{hyperref}

% --- Book appearance ---------------------------------------

\setstocksize{57em}{37em}
\settrimmedsize{57em}{37em}{*}
\setlrmarginsandblock{5em}{*}{1}
\setulmarginsandblock{5em}{*}{1}
\setlength{\headsep}{1.33em}
\setlength{\footskip}{2.5em}
\setlength{\parindent}{0em}
\setlength{\parskip}{0.6em}
\fixpdflayout
\checkandfixthelayout

\makepagestyle{thphp}
\makeevenhead{thphp}{\thepage}{\rightmark}{\thepage}
\makeoddhead{thphp}{\thepage}{\rightmark}{\thepage}
\makeoddfoot{thphp}{}{{}}{}
\makeevenfoot{thphp}{}{{}}{}
\pagestyle{thphp}

\renewcommand{\cftdot}{}
\setlength\cftparskip{1pt}

% --- Mathematical notation ---------------------------------

% Environments
\newenvironment{eqna}{\begin{IEEEeqnarray*}{c}}{\end{IEEEeqnarray*}\ignorespacesafterend}
\newenvironment{eqnb}{\begin{IEEEeqnarray*}{rCl}}{\end{IEEEeqnarray*}\ignorespacesafterend}
%\newcommand{\nimi}[1]{\IEEEyesnumber\label{#1}}
\renewenvironment{equation}{\sdfsdfsd}{\sdfsf}
\renewenvironment{align}{\sdfsfsd}{\sdfsd}

% Derivatives
\newcommand{\der}[2]{\frac{\dif#1}{\dif#2}}
\newcommand{\pder}[2]{\frac{\partial#1}{\partial#2}}

% Lorentz transformations
\newcommand{\lort}[2]{\Lambda^{#1}_{\phantom{#1}{#2}}}
\newcommand{\lori}[2]{\Lambda_{#1}^{\phantom{#1}{#2}}}

\newcommand{\chris}[3]{\{{_{#1}}{^{#2}}{_{#3}}\}}

% Miscellaneous
\newcommand{\puoli}{\frac{1}{2}}
\newcommand{\yksi}{\mathfrak{1}}
\newcommand{\andd}{\qquad\textrm{and}\qquad}
\newcommand{\wheree}{\qquad\textrm{where}\qquad}
\renewcommand{\vec}[1]{\mathbf{#1}}
\newcommand{\dvec}[1]{\dot{\vec{#1}}}
\newcommand{\ddvec}[1]{\ddot{\vec{#1}}}
\newcommand{\pvec}[1]{\primed{\vec{#1}}}
\DeclareMathOperator{\diag}{diag}
\DeclareMathOperator{\Det}{Det}
\DeclareMathOperator{\Tr}{Tr}
\DeclareMathOperator{\reaaliosa}{Re}
\DeclareMathOperator{\imaginaariosa}{Im}
\renewcommand{\Re}{\reaaliosa}
\renewcommand{\Im}{\imaginaariosa}
\newcommand{\dd}{\mathrm{d}}
\newcommand{\ii}{\mathrm{i}}
\newcommand{\ee}{\mathrm{e}}
\newcommand{\primed}[1]{\hat{#1}}
\newcommand{\paika}{\mathfrak{s}}
\newcommand{\ind}[1]{\mathfrak{#1}}
\newcommand{\circc}{\tau}
% UNCOMMENT NEXT LINE FOR TRADITIONAL CIRCLE CONSTANT 2 PI
%\renewcommand{\circc}{2\pi}

% Colors and indices
\definecolor{safi}{RGB}{0,100,100}
\definecolor{oranssi}{RGB}{255,128,0}
\newcommand{\coa}{{\color{black}\bullet}}
\newcommand{\cob}{{\color{oranssi}\blacklozenge}}
\newcommand{\coc}{{\color{cyan}\blacktriangleright}}
\newcommand{\cod}{{\color{red}\blacktriangleleft}}
\newcommand{\coe}{{\color{magenta}\blacktriangle}}
\newcommand{\cof}{{\color{green}\blacktriangledown}}
\newcommand{\cog}{{\color{safi}\bigstar}}
\newcommand{\coh}{{\color{yellow}\blacksquare}}
\newcommand{\tyh}{\phantom{\bullet}}
% UNCOMMENT NEXT LINE FOR TRADITIONAL GREEK INDICES
%\renewcommand{\coa}{\alpha}\renewcommand{\cob}{\beta}}\renewcommand{\coc}{\gamma}\renewcommand{\cod}{\delta}\renewcommand{\coe}{\mu}\renewcommand{\cof}{\nu}\renewcommand{\cog}{\rho}\renewcommand{\coh}{\sigma}\newcommand{\tyh}{\phantom{\alpha}}

% Dialog characters

\newcommand{\hea}{\(\blacklozenge\)\;}
\newcommand{\heb}{\(\Game\)\;}

\begin{document}
\frontmatter
\begin{titlingpage}
	\begin{centering}
		\HUGE\textbf{Time}\\
		\vspace{0.4em}
		\normalsize\emph{by}\\
		\vspace{0.4em}
		\textsc{Konsta Kurki}\\
		\vspace{0.4em}
		\textsc{\today}\\
		%\vspace{5em}
		%\vfill
		\vspace{5em}
		\textbf{---work in progress---}\\
	\end{centering}
	\vfill

	Copyright {\textcopyright} 2015 Konsta Kustaa Kurki

	This work is licensed under a Creative Commons Attribution 4.0 International License. See \url{http://creativecommons.org/licenses/by/4.0/} for the license and \url{http://github.com/konstakurki/time} for the source material.
\end{titlingpage}
\chapter{Change log}
\textsc{June 28, 2015}\, Initial commit to \url{github.com/konstakurki/time}
\mainmatter
\chapter{The beginning}
\hea Excuse me.

\heb Yes?

\hea Could you tell me about time?

\heb Hmm. You mean the stuff you read from the upper right corner of your iPhone?

\hea I do not have an iPhone. But if I had I think it would be that.

\heb Why would you like to hear about it?

\hea I know how to read a clock but that's where my understanding ends and wondering begins. Sometimes I think I flow in time. Next moment I think time flows around me. And then I think nothing flows.

\heb I certainly feel you! Time's a peculiar fellow. Did you notice what kind of words you just used? Like `sometimes' and `moment'.

\hea Yeah, I know, it's embarrassing.

\heb You cannot escape him! Or her.

\hea I know! I just don't know how to communicate without referring to time. Does it mean that I don't understand my own words?

\heb Well, it may be so. But don't worry---you're no worse than any one of us.

\hea You mean there's no person on Earth who understands his own words?

\heb I think understanding is just a feeling. Peace, acceptance. But it may be just a flash in time. In \emph{time!}

\hea For me it is not so easy to accept things like the fact that now it is a different moment than a couple of minutes ago. And now it is again different.

\heb It surely is. Do you understand that there is no brief answer?

\hea I think I can take a long one.

\heb You also think I would give you such?

\hea Wouldn't you?

\heb There's something in your eyes which I haven't seen before. I cannot be sure but I feel that you might really want to hear it. Am I right or are you just hoaxing me?

\hea I would really, really want to hear. I want it really badly.

\heb All right. But our path will be very, very long. And we probably won't even get any final answer. That's because I haven't found it. As far as I know no one has. All we can do is just wander around and build some theory. Beautiful, but imperfect theory. Can you accept that?

\hea Just give me something. I must get somewhere with this question. Please. I've asked from many people and all they say is something really silly like 'time does not exist'. If something makes me that anxious it definitely must be there!

\heb I like you. I can feel the pain inside you with all my neurons. Not that I liked that someone does not feel good, but you seem to care about things that people usually do not even notice.

\heb Do you have any background?

\hea Acually I do. I've studied foundations of mechanics. I know Newton's laws, the action principle and Hamilton's equations. And I also know some quantum mechanics.

\heb Good! Then we can fly over some boring stuff.

\section{Sharpening the question}
\heb Tell me some important things of your life.

\hea My family. And friends. Freedom.

\heb Something more primitive.

\hea Hmmm \ldots birth.

\heb Excellent! The primitive elements we use in construing our lives are events. How would you order your birth, day you learned to speak and the cutting of your umbilical cord?

\hea I guess I could order them in many ways. Of course there is the one way of ordering them, but probably you don't mean that.

\heb Very often the right direction is so obvious and ridiculously simple that you do not even dear to think of it. So go on.

\hea Ok. So first I was born and then my umbilical cord were cut. Several years after those things I said my first word. Am I right?

\heb Yes you are! What I mean by `time' is the affair---whatever it is in its deepest---that creates this order. Or at least I don't know how to describe it more precisely. Do you agree?

\hea I guess I do. We can measure it easily, right? For example by whatching the corner of a stupid iPhone you mentioned.

\heb Yes we can. But there is some pretty complicated technology inside that device. There are also simpler methods.

\hea Of course. I think people have investigated how much time there is between two events for generations by counting for example how many sunsets fits in between the events. I guess any repeating phenomenon would do it.

\heb That is indeed how each and every clock work. The `time that firs in between the events' is called time interval of the events and is usually denoted by \(\Delta t\), the big triangle, the Greek capital delta meaning interval and \(t\) time.

\heb If we count sunsets we get a bigger number than if we counted for example summers. But what is beautiful is that the relation between the frequencies of sunsets and summers never changes! It is always something like 365. If we take any two good clocks, their relative paces remain always the same.

\hea That definitely makes sense. If your clock has same kind of second today than mine, I assume with no though that same is true also tomorrow. Does this mean that the time interval of two events is in some sense an absolute concept?

\heb Up to the time scale fixed by the clock, yes. We can of course convert time intervals read from different clocks to each other by knowing the relative paces of the clocks.

\heb It is useful to choose some reference time interval, for example day which is the interval of two successive midnights, or the second you mentioned, and express all intervals as multiples of it. Then each and every human being get the same number of days for the same two events.

\hea So the time interval of two arbitrary events measured in days---or in seconds---is an absolute, observer--independent number. Sounds very reasonable.

\hea But wait a second. I watched \emph{Interstellar} a couple of weeks ago. In one scene the main character Cooper gives a wrist watch to his daughter Murph and explains that when he returns from his space voyage and they compare their watches, they will find out that the clocks are not anymore in sync, even though the clocks are identical. Was it crap or what?

\heb No it wasn't. Cooper is referring to a concept called proper time. It means the personal time experienced by an observer which in fact is different for Cooper and Murph. But Cooper flies in very exotic circumstances. Here, on Earth where we cannot move as fast as Cooper and the proper times of us all are pretty much the same. That effectively universal proper time is called Galilean time and is the time we have been talking about.

\hea So now you tell me that the time interval were not absolute. Why did you try to lie to me?

\heb In our everyday regime it really is absolute. We do not notice any deviations, and for hundreds of years people really believed in its strick absoluteness. If you are patient, we will get to the relativity. Right now it would probably be too complicated to think of time intervals as subjective, but later we will see that actually the relativistic picture is astonishingly simple and beautiful.

\hea Okay. I accept that.
\subsection{Space}
\heb Could we measure intervals of some other physical, perhaps even more concrete, thing than time?

\hea You must mean distances in space.

\heb Yes I do. If you were about to measure say the distance between two stones, how would you do that?

\hea I would take a rope, tie some knots equidistantly on it, tighten it between the stones and count the knots in between.

\heb Excellent! I would have taken a meter stick but this is much more elegant. The process is very similar to measuring time; here the rope plays the role of a clock and knot spacing sets up a distance scale.

\hea And like time intervals, also spatial distances are absolute up to a distance scale!

\heb Yes, at least in the regime of everyday phenomena.

\heb Space is very much like time, but there is a difference: in time you can only imagine to go back and forward; in space you can also go left, right up and down.

\hea This sounds a bit stupid to me. I think that in time I cannot move back nor forward; time just passes. In space I can move, but it takes some time. I mean that `moving' means that location changes with time. It is ridiculous to say that time changes with time.

\heb Yeah, yeah. That's why I said that you can imagine. It is somewhat sloppy language. But the point is that space is three--dimensional while time has only one dimension.

\hea Yes. That's a difference.
\subsection{Perspectives}
\heb Let us consider some observers like you and me. Let's assume that the observers do not move and they all use same units of time and distance. How can these observers differ from each other?

\hea Well, I think everybody is different. Even if two persons were cloned from the same DNA, they have necessarily faced different things in life and become different kind of persons.

\heb Very clever. I wouldn't have known that! But you know that I meant something simpler.

\hea The observers are located at different places.

\heb And their noses may point in different directions. It is interesting to think of how the perspectives of these different observers are related to each other.

\heb We human beings have two eyes and therefore a stereo vision. We can locate things accurately in space by just watching. Say there is a couple of stones around and every observer see them. The question is: how are their points of view, I mean literally the image that gets projected on their verkkokalvo, relate to each other?

\hea Sounds like a complicated question.

\heb Well, the actual transformation of the view may be complicated, but we can say something about them. Let us denote the image on the verkkokalvos of some observer by \(\ket{\Psi}\). This notation is from quantum mechanics; I chose it because it looks so cool. Here it only means the view and nothing quantum mechanical.

\heb If we for example translate the viewer, that is change her location, the view changes. Let us denote the translation by \(\vec{T}\) and write
\begin{eqna}
    \ket{\Psi}\rightarrow\vec{T}\ket{\Psi}.
\end{eqna}

\hea Sounds trivial.

\heb I know. Now we can make another translation \(\primed{\vec{T}}\). In totality we have
\begin{eqna}
    \ket{\Psi}\rightarrow\vec{T}\ket{\Psi}\rightarrow\primed{\vec{T}}\vec{T}\ket{\Psi}.
\end{eqna}
Clearly both translations together form another translation which we may think of as a product \(\primed{\vec{T}}\vec{T}\). We can also think of doing nothing as a special kind of translation and denote it by \(1\).

\heb This is to say that mathematically the translations form a group. It is clear that the order of translations does not matter; we have \(\primed{\vec{T}}\vec{T}=\vec{T}\primed{\vec{T}}\) for any two translations. Translations are said to commute.

\hea I've heard of groups. If we translate by \(T\) we can always translate back, which I guess we could denote by \(T^{-1}\). It holds that \((\primed{T}T)^{-1}=T^{-1}\primed{T}^{-1}\) because
\begin{eqna}
    T^{-1}\primed{T}^{-1}\primed{T}T=T^{-1}1T=1.
\end{eqna}
This is clear because translating and then translating back is doing nothing. I feel stupid when I say that out loud.

\heb Good. Things get more interesting when we consider rotations. Rotations also clearly form a group which in three dimensions is known as \(SO(3)\).

\heb The group of rotations is more interesting because it is not commutative. Look: If I take this chair and do two \(\circc/4\) rotations around vertical and horizontal axes, I get different results depending on the order.

\hea I see, but what is \(\circc\)?

\heb It is the circle constant \(\circc\doteq2\pi\). \(\pi\) is used only for historical reasons; \(\circc\) is much more natural as it represens the whole circle. Check \url{tauday.com} for exhaustive discussion. If you don't like this, just go to the sourcecode of our lives and uncomment one line. The source can be found at \url{konstakurki.co/time}.

\hea Cool.
\end{document}
