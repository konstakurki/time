\documentclass[10pt,oneside%,draft%
]{memoir}

% --- Packages ----------------------------------------------

\usepackage[USenglish]{babel}
\usepackage[utf8]{inputenc}
\usepackage[T1]{fontenc}
\usepackage{textcomp}
\usepackage{color}
\usepackage{graphicx}
\usepackage{IEEEtrantools}
\usepackage{verbatim}
\usepackage{tocloft}
\usepackage{amsmath}
\usepackage{amsfonts}
\usepackage{amssymb}
\usepackage{braket}
\usepackage[hyphens]{url}
\usepackage{makeidx}
\usepackage[colorlinks=true,urlcolor=blue,linkcolor=blue,linktocpage=true]{hyperref}

% --- Book appearance ---------------------------------------

\setstocksize{57em}{37em}
\settrimmedsize{57em}{37em}{*}
\setlrmarginsandblock{5em}{*}{1}
\setulmarginsandblock{5em}{*}{1}
\setlength{\headsep}{1.33em}
\setlength{\footskip}{2.5em}
\setlength{\parindent}{0em}
\setlength{\parskip}{0.6em}
\fixpdflayout
\checkandfixthelayout

\makepagestyle{thphp}
\makeevenhead{thphp}{\thepage}{\rightmark}{\thepage}
\makeoddhead{thphp}{\thepage}{\rightmark}{\thepage}
\makeoddfoot{thphp}{}{{}}{}
\makeevenfoot{thphp}{}{{}}{}
\pagestyle{thphp}

\renewcommand{\cftdot}{}
\setlength\cftparskip{1pt}

% --- Mathematical notation ---------------------------------

% Environments
\newenvironment{eqna}{\begin{IEEEeqnarray*}{c}}{\end{IEEEeqnarray*}\ignorespacesafterend}
\newenvironment{eqnb}{\begin{IEEEeqnarray*}{rCl}}{\end{IEEEeqnarray*}\ignorespacesafterend}
\newenvironment{narration}{\begin{em}}{\end{em}}
\newcommand{\nimi}[1]{\IEEEyesnumber\label{#1}}
\renewenvironment{equation}{\sdfsdfsd}{\sdfsf}
\renewenvironment{align}{\sdfsfsd}{\sdfsd}

% Derivatives
\newcommand{\der}[2]{\frac{\dd#1}{\dd#2}}
\newcommand{\pder}[2]{\frac{\partial#1}{\partial#2}}
\newcommand{\cder}[2]{\frac{D #1}{D #2}}

% Symbols
\newcommand{\puoli}{\frac{1}{2}}
\newcommand{\yksi}{\mathfrak{1}}
\newcommand{\andd}{\qquad\textrm{and}\qquad}
\newcommand{\orr}{\qquad\textrm{or}\qquad}
\newcommand{\wheree}{\qquad\textrm{where}\qquad}
\newcommand{\dd}{\mathrm{d}}
\newcommand{\ii}{\mathrm{i}}
\newcommand{\ee}{\mathrm{e}}
\newcommand{\circc}{\tau}
\newcommand{\paika}{\mathfrak{s}}


\renewcommand{\vec}[1]{\mathbf{#1}}
\newcommand{\dvec}[1]{\dot{\vec{#1}}}
\newcommand{\ddvec}[1]{\ddot{\vec{#1}}}
\newcommand{\pvec}[1]{\primed{\vec{#1}}}

% Operators
\DeclareMathOperator{\diag}{diag}
\DeclareMathOperator{\Det}{Det}
\DeclareMathOperator{\Tr}{Tr}
\DeclareMathOperator{\reaaliosa}{Re}
\DeclareMathOperator{\imaginaariosa}{Im}
\renewcommand{\Re}{\reaaliosa}
\renewcommand{\Im}{\imaginaariosa}

\newcommand{\primed}[1]{\hat{#1}}
\newcommand{\ind}[1]{\mathfrak{#1}}
\newcommand{\arxivreference}[1]{\url{#1}}

% Differential geometry
\newcommand{\chris}[3]{\{{_{#1}}\!{^{#2}}\!{_{#3}}\}}
\newcommand{\tensy}[2]{#1^{#2}}
\newcommand{\tensa}[2]{#1_{#2}}
\newcommand{\tensay}[3]{#1_{#2}^{\phantom{#2}#3}}
\newcommand{\tensya}[3]{#1^{#2}_{\phantom{#2}#3}}

% Dialog characters
\newcommand{\hea}{\(\blacklozenge\)\;}
\newcommand{\heb}{\(\Game\)\;}

% Colors and indices
\definecolor{safi}{RGB}{0,100,100}
\definecolor{oranssi}{RGB}{255,128,0}
\newcommand{\coa}{{\color{black}\bullet}}
\newcommand{\cob}{{\color{oranssi}\bullet}}
\newcommand{\coc}{{\color{cyan}\bullet}}
\newcommand{\cod}{{\color{red}\bullet}}
\newcommand{\coe}{{\color{magenta}\bullet}}
\newcommand{\cof}{{\color{green}\bullet}}
\newcommand{\cog}{{\color{safi}\bullet}}
\newcommand{\coh}{{\color{yellow}\bullet}}
\newcommand{\coi}{{\color{blue}\bullet}}

% UNCOMMENT THE NEXT LINE FOR TRADITIONAL GREEK INDICES
%\renewcommand{\coa}{\alpha}\renewcommand{\cob}{\beta}\renewcommand{\coc}{\gamma}\renewcommand{\cod}{\delta}\renewcommand{\coe}{\mu}\renewcommand{\cof}{\nu}\renewcommand{\cog}{\rho}\renewcommand{\coh}{\sigma}\renewcommand{\coi}{\xi}

% UNCOMMENT THE NEXT LINE FOR THE TRADITIONAL CIRCLE CONSTANT 2 PI
%\renewcommand{\circc}{2\pi}

\begin{document}
\frontmatter
\begin{titlingpage}
	\begin{centering}
		\HUGE\textbf{Time}\\
		\vspace{0.4em}
		\normalsize\emph{by}\\
		\vspace{0.4em}
		\textsc{Konsta Kurki}\\
		\vspace{0.4em}
		\textsc{\today}\\
		%\vspace{5em}
		%\vfill
		\vspace{5em}
		\textbf{---work in progress---}\\
	\end{centering}
	\vfill

	Copyright {\textcopyright} 2015 Konsta Kustaa Kurki

	This work is licensed under a Creative Commons Attribution 4.0 International License. See \url{http://creativecommons.org/licenses/by/4.0/} for the license and \url{http://github.com/konstakurki/time} for the source material.
\end{titlingpage}
\chapter{Change log}
\begin{narration}
This is an incomplete list of additions and changes made since the beginning of the writing.
\end{narration}


\textsc{June 28, 2015}\, Initial commit to \url{github.com/konstakurki/time}

\newpage
\tableofcontents
\mainmatter
\book{The classical picture}
\part{The very elements}
\chapter{The beginning}
\hea Excuse me.

\heb Yes?

\hea Could you tell me about time?

\heb Hmm. You mean the stuff you read from the upper right corner of your iPhone?

\hea I do not have an iPhone. But if I had I think it would be that.

\heb Why would you like to hear about it?

\hea I know how to read a clock but that's where my understanding ends and wondering begins. Sometimes I think I flow in time. Next moment I think time flows around me. And then I think nothing flows.

\heb I certainly feel you! Time's a peculiar fellow. Did you notice what kind of words you just used? Like `sometimes' and `moment'.

\hea Yeah, I know, it's embarrassing.

\heb You cannot escape him! Or her.

\hea I know! I just don't know how to communicate without referring to time. Does it mean that I don't understand my own words?

\heb Well, it may be so. But don't worry---you're no worse than any one of us.

\hea You mean there's no person on Earth who understands his own words?

\heb I think understanding is just a feeling. Peace, acceptance. But it may be just a flash in time. In \emph{time!}

\hea For me it is not so easy to accept things like the fact that now it is a different moment than a couple of minutes ago. And now it is again different.

\heb It surely is. Do you understand that there is no brief answer?

\hea I think I can take a long one.

\heb You also think I would give you such?

\hea Wouldn't you?

\heb There's something in your eyes which I haven't seen before. I cannot be sure but I feel that you might really want to hear it. Am I right or are you just hoaxing me?

\hea I would really, really want to hear. I want it really badly.

\heb All right. But our path will be very, very long. And we probably won't even get any final answer. That's because I haven't found it. As far as I know no one has. All we can do is just wander around and build some theory. Beautiful, but imperfect theory. Can you accept that?

\hea Just give me something. I must get somewhere with this question. Please. I've asked from many people and all they say is something really silly like 'time does not exist'. If something makes me that anxious it definitely must be there!

\heb I like you. I can feel the pain inside you with all my neurons. Not that I liked that someone does not feel good, but you seem to care about things that people usually do not even notice.

\heb Do you have any background?

\hea Acually I do. I've studied foundations of mechanics. I know Newton's laws, the action principle and Hamilton's equations. And I also know some quantum mechanics.

\heb Good! Then we can fly over some boring stuff.
\subsection{Sharpening the question}
\heb Tell me some important things of your life.

\hea My family. And friends. Freedom.

\heb Something more primitive.

\hea Hmmm \ldots birth.

\heb Excellent! The primitive elements we use in construing our lives are events. How would you order your birth, day you learned to speak and the cutting of your umbilical cord?

\hea I guess I could order them in many ways. Of course there is the one way of ordering them, but probably you don't mean that.

\heb Very often the right direction is so obvious and ridiculously simple that you do not even dear to think of it. So go on.

\hea Ok. So first I was born and then my umbilical cord were cut. Several years after those things I said my first word. Am I right?

\heb Yes you are! What I mean by `time' is the affair---whatever it is in its deepest---that creates this order. Or at least I don't know how to describe it more precisely. Do you agree?

\hea I guess I do. We can measure it easily, right? For example by whatching the corner of a stupid iPhone you mentioned.

\heb Yes we can. But there is some pretty complicated technology inside that device. There are also simpler methods.

\hea Of course. I think people have investigated how much time there is between two events for generations by counting for example how many sunsets fits in between the events. I guess any repeating phenomenon would do it.

\heb That is indeed how each and every clock work. The `time that firs in between the events' is called time interval of the events and is usually denoted by \(\Delta t\), the big triangle, the Greek capital delta meaning interval and \(t\) time.

\heb If we count sunsets we get a bigger number than if we counted for example summers. But what is beautiful is that the relation between the frequencies of sunsets and summers never changes! It is always something like 365. If we take any two good clocks, their relative paces remain always the same.

\hea That definitely makes sense. If your clock has same kind of second today than mine, I assume with no though that same is true also tomorrow. Does this mean that the time interval of two events is in some sense an absolute concept?

\heb Up to the time scale fixed by the clock, yes. We can of course convert time intervals read from different clocks to each other by knowing the relative paces of the clocks.

\heb It is useful to choose some reference time interval, for example day which is the interval of two successive midnights, or the second you mentioned, and express all intervals as multiples of it. Then each and every human being get the same number of days for the same two events.

\hea So the time interval of two arbitrary events measured in days---or in seconds---is an absolute, observer--independent number. Sounds very reasonable.

\hea But wait a second. I watched \emph{Interstellar} a couple of weeks ago. In one scene the main character Cooper gives a wrist watch to his daughter Murph and explains that when he returns from his space voyage and they compare their watches, they will find out that the clocks are not anymore in sync, even though the clocks are identical. Was it crap or what?

\heb No it wasn't. Cooper is referring to a concept called proper time. It means the personal time experienced by an observer which in fact is different for Cooper and Murph. But Cooper flies in very exotic circumstances. Here, on Earth where we cannot move as fast as Cooper and the proper times of us all are pretty much the same. That effectively universal proper time is called Galilean time and is the time we have been talking about.

\hea So now you tell me that the time interval were not absolute. Why did you try to lie to me?

\heb In our everyday regime it really is absolute. We do not notice any deviations, and for hundreds of years people really believed in its strick absoluteness. If you are patient, we will get to the relativity. Right now it would probably be too complicated to think of time intervals as subjective, but later we will see that actually the relativistic picture is astonishingly simple and beautiful.

\hea Okay. I accept that.
\section{Space}
\heb Could we measure intervals of some other physical, perhaps even more concrete, thing than time?

\hea You must mean distances in space.

\heb Yes I do. If you were about to measure say the distance between two stones, how would you do that?

\hea I would take a rope, tie some knots equidistantly on it, tighten it between the stones and count the knots in between.

\heb Excellent! I would have taken a meter stick but this is much more elegant. The process is very similar to measuring time; here the rope plays the role of a clock and knot spacing sets up a distance scale.

\hea And like time intervals, also spatial distances are absolute up to a distance scale!

\heb Yes, at least in the regime of everyday phenomena.

\heb Space is very much like time, but there is a difference: in time you can only imagine to go backwards and forward; in space you can also go left, right up and down.

\hea This sounds a bit stupid to me. I think that in time I cannot move back nor forward; time just passes. In space I can move, but it takes some time. I mean that `moving' means that location changes with time. It is ridiculous to say that time changes with time.

\heb Yeah, yeah. That's why I said that you can imagine. It is somewhat sloppy language. But the point is that space is three--dimensional while time has only one dimension.

\hea Yes. That's a difference.
\subsection{Some properties}
\heb What is the shortest path between to points in space?

\hea If I tighten a rope between the points, it of course settles down to a configuration in which there are as little rope as possible between the points. That's what tightening means. So the path the rope takes is the shortest path between the two points.

\hea And, of course, the path is also the straight path between the points. The straight path is the shortest and \emph{vice versa}.

\heb Excellent. This is one of the elementary properties of our space. The triangle inequality of elementary geometry can be seen as a special case of this fact.

\heb Think of some object moving in space, for example a baseball. Let us follow its trajectory for a little time \(\Delta t\). What can you say about the trajectory?

\hea Well, it is some curve going through space. We can think of it as having a direction pointing to the direction ???

\heb True. But if we now decrease \(\Delta t\) down to almost vanishing time interval \(\dd t\), we get only a small portion of the curve which looks practically straight. Let me denote that small arrow by \(\dd x\).

\heb Such an arrow is called a vector. A vector is characterized by its length and the direction it points to. Our first vector \(\dd x\) can be thought of as being localized at somewhere on the trajectory of the baseball, but right now it is not important.

\hea Now we can divide \(\dd x\) by \(\dd t\) and get another vector
\begin{eqna}
    \der{x}{t}\doteq\dot{x},
\end{eqna}
the velocity. If we used \(\dd t/2\) instead of \(\dd t\), the particle would have made only half of \(\dd x\). The velocity does not depend on the particular value of \(\dd t\) if it is small enough.

\hea Why the location does not matter?

\heb It is because we can always parallel transport, that is, move the vector while keeping its length and direction unchanged. You can do this for example with the aid of a compass.

\hea Wait a minute. If I go around the magnetic north pole, the needle of the compass turns around. I wouldn't really trust on a compass.

\heb True. But there is also another elegant and physically interesting way of doing the parallel transport.

\heb Take a wheel of a bicycle and put it to spin. You will notice that the spinning makes the it difficult to change the direction of the axis. With fast enough spin and hyva ripustus the wheel can be used to keep track of a direction.

\hea Oh, you are talking about a gyroscope. They're interesting devices.

\heb We can add two vectors by moving, of course by parallel transport, the other arrow starting from the end of the other. The sum is the arrow starting from the beginning of the first vector and ending at the end of the second. Vectors can also be multiplied by numbers. The properties of vector spaces are very intuitive, but if you miss some mathematical details, check for example the methodology book by Riley, Hobson and Bence?.
\subsection{Tensors}
\heb We can consider functions of vectors. A very useful class of functions are functions that preserve the algebraic structure of vector space.

\hea You mean addition and scalar multiplication? 

\heb Yes. Such a function is said to be linear. Say we have vectors \(A\) and \(B\) and numbers \(a\) and \(b\). For a linear function \(f\) it holds that
\begin{eqna}
    f(aA+bB)=af(A)+bf(B).
\end{eqna}
Linear functions are very simple and intuitive, for example zero vector gets always mapped to zero as can be seen from the equation above after choosing \(a=b=0\).
%visuaalinen oneformin esiitys

\heb That kind of linear functions can also be thought of vectors of another vector space called the dual vector space. The sum of dual vectors \(f\) and \(g\) means that for any vector \(V\) and number \(n\) it holds that
\begin{eqna}
    (f+g)(V)=f(V)+g(V)
\end{eqna}
and
\begin{eqna}
    (nf)(V)=nf(V).
\end{eqna}

\hea That's pretty reasonable and starightforward.

\heb We can also consider linear maps, or tensors as they are usually called, which take any number of vectors as inputs and are linear in every input vector.

\heb The notation becomes a bit complex if a tensor has many inputs. That is why I think we should adopt a funny but practical notation. If we have for example a tensor \(T\) which takes three vectors and the input vectors are \(A\), \(B\) and \(C\), we could write
\begin{eqna}
    T(A,B,)\doteq\tensa{T}{\coa\cob\coc}A^\coa B^\cob C^\coc.
\end{eqna}

\hea Looks hilarious. What are those colored balls?

\heb They are wireless connectors that plug the vectors to the tensor. People usually prefer greek letters like \(\alpha\) and \(\mu\). If you would like to see letters instead of colors, you must go to the source code of our lives and uncomment one line.

\hea I like color.

\heb Me too! The order of writing of the vectors and the tensor does not matter, the colors keep track of which connector, or index as they are often called, is connected to which. A single vector \(V\) mapped by a dual vector \(f\) can be therefore written as
\begin{eqna}
    f_\coa V^\coa=V^\coa f_\coa.
\end{eqna}
This suggests that we can think of \(V^\coa\) as a tensor which takes the dual vector \(f_\coa\) as its input! It does not really matter how we choose to think of it; we just connect an upper index to a lower and that's it.

\hea So we could have tensors with lower and upper indices, for example \(\tensya{K}{\coa\cob\coc}{\cod\coe\cof}\), taking vectors and dual vectors as inputs?

\heb Yes, we very well could. And as you probably guess, we can write any number of tensors in a row and then connect, or contract as it is often said, any lower index with any upper one.

\hea Actually it is useful to firs think of forming an outer product, for example
\begin{eqna}
    \tensya{(KT)}{\coa\cob\coc}{\cod\coe\cof\cog\coh\coi}=\tensya{K}{\coa\cob\coc}{\cod\coe\cof}\tensa{T}{\cog\coh\coi}.
\end{eqna}
It is a tensor which takes vectors and dual vectors just as you would guess from the indices. It is easy to verify that it is linear in every index. Then we contract any upper index--lower index pairs we want. We can for example contract all three pairs of the above outer product and get
\begin{eqna}
    \tensya{(KT)}{\coa\cob\coc}{\cod\coe\cof\coa\cob\coc}\doteq\tensa{(KT)}{\cod\coe\cof}
\end{eqna}
which is a tensor with three lower indices.

\hea How elegant is that! I just take some tensors, form an outer prodct of them by writing them in a row in whichever order I want, and then contract the indices I want. Every contraction removes one upper and one lower index. If there are equal numbers of upper and lower indices and I contract them all, I am left with a number.

\heb It is just that simple. Vectors are just tensors with one upper index and dual vectors tensors with one lower index. A number can be thought of as a tensor with no indices at all.

\hea This is very, very cool. I think I'm gonna read something about vector spaces and tensors.
\subsection{Inner product}
\hea I read that a vector space is often equipped with an inner product. An inner product of two vectors is linear and symmetric in both of the vectors. If it is a linear map which takes two vectors, say \(V\) and \(W\), I think we can write it as
\begin{eqna}
    V\cdot W\doteq g_{\coa\cob}V^\coa W^\cob
\end{eqna}
with \(g_{\coa\cob}\) a tensor.

\heb Great! Because the inner product is symmetric, we have \(g_{\coa\cob}=g_{\cob\coa}\). This may seem a bit sloppy, but it means that
\begin{eqna}
    V\cdot W\doteq g_{\coa\cob}V^\coa W^\cob=V\cdot W\doteq g_{\cob\coa}V^\coa W^\cob
\end{eqna}
for any \(V^\coa\) and \(W^\cob\).

\heb Inner product describes the overlap of two vectors. If consider the overlap of a vector, for example a separation \(\dd x\), with itself, we get a number wich is proportional to the square of the length of \(\dd x\).

\hea So what you are trying to say is that we choose \(g_{\coa\cob}\) so that
\begin{eqna}
    |\dd x|^2=g_{\coa\cob}\dd x^\coa\dd x^\cob
\end{eqna}
exactly?

\heb Yes. The tensor \(g\) is called the metric tensor of space, since it measures the distance separated by \(\dd x\). For \(\dd x\) one usually writes \(|\dd x|^2\doteq\dd s^2\) and for any other vector \(V\) that \(|V|^2\doteq V^2\). It is a bit inelegant to have different conventions for different vectors but this notation is standard and widespread.
\subsection{Rising and lowering}
\heb The metric is so important that for each vector \(V^\coa\) we define a dual vector \(g_{\coa\cob}V_\cob\doteq V_\coa\). That is, we really do not bother to think vectors and dual vectors as different kinds of things; we only have vectors which can be written the index in the upstairs or the downstairs.

\heb We can actually lower any upper index of any tensor in the same way, for example \(g_{\coa\cob}T^{\cob\coc}\doteq\tensay{T}{\coa}{\coc}\).

\hea You said that we don't bother to fundamentally distinguish vectors and dual vectors. So we can raise an index back up as easily as we lowered it?

\heb Yes we can. For this purpose we define an inverse metric \(g^{\coa\cob}\) by
\begin{eqna}
    g^{\coa\cob}g_{\cob\coc}=1^{\coa}_{\coc},
\end{eqna}
in which \(1\) is the unit tensor which does nothing. We could also think of \(1\) as the metric \(g\) with the other index rised. So first lowering and then raising is doing nothing as it should be. And the same holds of course for first rising and then lowering.

\hea So the picture simplifies even more. We write tensors with indices placed where ever it is practical and contract. The metric never needs to be written explicitly. Damn this is elegant.

\heb Yeh, however, of course we may sometimes want to write it visible, for example when I write
\begin{eqna}
    \dd s^2=g_{\coa\cob}\dd x^\coa\dd x^\cob
\end{eqna}
I want to emphasize that the metric defines the distance of two nearby points.

\heb Also remember that rising and lowering, and therefore cavalry? about the true position of an index, were made possible by the existence of the metric. In some other situations there may not be such a distinguished metric and then we may really need to distinguish vectors and dual vectors. This is pretty abstract, but will be highly useful for us in the future.
\subsection{Length of a curve}
\heb Now we can calculate the length of any path or curve in space by
\begin{eqna}
    s=\int\dd s=\int\sqrt{g_{\coa\cob}\dd x^\coa\dd x^\cob}.
\end{eqna}
We can also parametrize the path for example by the elapsed time \(t\) when a body moves through the path. Then we have
\begin{eqna}
    s=\int\sqrt{g_{\coa\cob}\der{x^\coa}{t}\der{x^\cob}{t}}\,\dd t=\int\sqrt{g_{\coa\cob}\dot{x}^\coa\dot{x}^\cob}\,\dd t.
\end{eqna}
\subsection{Differentiating tensors}
\heb Vectors can be added together and multiplied by numbers, so we can differentiate a vector \(V\) that depends for example on \(t\) easily,
\begin{eqna}
    \dot{V}=\frac{V(t+\dd t)-V(t)}{\dd t}=\der{V}{t},
\end{eqna}
and the result is another vector. The variable \(t\) do not need to mean time; it can also mean for example the position on some curve in space, if there is a vector on every point on the curve. See, here it is useful to talk about vectors localized somewhere.

\heb We can do the same for any tensor. Like for almost any product, the Leibnitz rule holds also for outer products of tensors. For example
\begin{eqnb}
    \der{}{t}(V\cdot W)&=&\der{}{t}\left(g_{\coa\cob}V^\coa W^\cob\right)\\
                       &=&\dot{g}_{\coa\cob}V^\coa W^\cob+g_{\coa\cob}\dot{V}^\coa W^\cob+g_{\coa\cob} V^\coa\dot{W}^\cob.
\end{eqnb}
If \(V\) and \(W\) are any unchanging vectors, also their inner product must remain unchanged. Therefore \(\dot{g}\equiv0\). The metric tensor is constant---a highly plausible result.

\heb Because the metric is constant, it holds that
\begin{eqna}
    \der{}{t}g_{\coa\cob}V^\cob=g_{\coa\cob}\dot{V}^\cob,
\end{eqna}
that is, the contraction of a derivative is the derivative of a contraction. This makes thing even more easy: we can just think of the derivative of a vector---or any other tensor---indices placed anywhere we want.
\section{Perspectives}
\heb Things don't look the same from the perspectives of different persons, right?

\hea Definitely not. A capitalist feels envy for a poor guy getting an unemployment compensation. The capitalist thinks that the poor guy hasn't done anything to earn it. On the other hand the poor guy thinks that the capitalist is acting selfishly since he has so much money that he could save many from starvation but is not willing to do that.

\heb Yeah, that's so very absurd. Even though we have all the technology to give everyone a good life and make hating old--fashioned, we're not willing to pursue that opportunity.

\heb But there are also another kinds of differences of perspectives that are more relevant for as here.

\hea I know, I just wanted to make an important remark. The motion of a rollercoaster looks different when observed from different locations.

\heb Also the look changes if the observer turns his eyes and looks to different locations. But we understand space as a somewhat objective thing and can think of things like locations and vectors without choosing any particular observer postioned and orientated in a particular way, right?

\hea Yeah.

\heb But there is an affair that matters.

\hea What is it?

\heb Motion. The differences in the motions of observers cannot be dealt so easily as the differences in positions and orientations.

\hea Why is that? We just have to take into account that locations change due time. If I have two points and their separation \(\dd x\), a moving observer sees both points moving and understans that the separation \(\dd x\) does not change at all. Both observers can understand it as the same separation vector. I would like to that think vectors and tensor are the same even though people may move.

\heb Well, for the separation vector of two fixed points, it really is so. Also the metric, which represents the overlap of two vectors can be seen as the same, because the overlap of two vectors must remain unchanged if the vectors itself remain unchanged.

\heb But think about the velocity of a particle. One observer sees its position changing by \(\dd x\) in a time interval \(\dd t\). But another observer, moving with respect to the first, sees a separation \(\dd\primed{x}=\dd x+\dd o\) in which \(\dd o\) is the motion of the first observer from the perspective of the firs. For the velocity vectors \(\dot{x}\) and \(\dot{\primed{x}}\) it holds that
\begin{eqna}
    \dot{\primed{x}}=\der{\primed{x}}{t}=\der{x+\dd o}{t}=\dot{x}+\dot{o}.
\end{eqna}
We cannot think of the velocity vector objectively, unless we assume that there is some natural state of `resting'.

\hea Do we?

\heb Aristotle? and co. assumed that natural resting is not moving with respect to Earth.

\heb But think about some simple phenomena, for example a swinging pendulum. Does the swinging care about motion?

\hea Well, if I put the pendulum on top of a car the air probably disturbs it, but yeah, I get the point. Inside the car it swings just like inside a house standing firmly on rock, at least if the ride is not so bumpy.

\heb Correct. If we eliminate disturbances like wind, the pendulum siwings the same way always when the lab is moving with a constant velocity. A house, a uniformly moving car, an elevator are all equally good labs.

\heb The idea that there is a natural equality of observers that move uniformly with respect to each other is called the principle of relativity. It seems to be very generally true.

\heb However, accelerating observers, I mean those that are moving nonuniformly with each other, are not equal. Pendulum swings differently in carousel and in a car driven by a mad man.

\hea Interesting. And sensible.

\heb If two observers move uniformly, with a constant velocity \(\dot{o}\) with respeect to each other, we have \(\dot{\primed{x}}=\dot{x}+\dot{o}\) as we noted. But for the acceleration \(\ddot{x}\) it holds that
\begin{eqna}
    \ddot{\primed{x}}=\der{}{t}\dot{\primed{x}}=\der{}{t}(\dot{x}+\dot{o})=\ddot{x}
\end{eqna}
since \(\ddot{o}\equiv 0\). The principle of relativity could be stated as `Acceleration is absolute, velocity is not.' The principle of relativity could be formulated by saying that \emph{acceleration is absolute and objective, velocity is not.}
\subsection{Newton's first law}
\heb Now we are ready to formulate an equation describing a law of nature.

\hea Cool!

\heb Consider a body, for example a rock, isolated as well as we can from the rest of the world. For simplicity let us neglect the possible rotation of the rock, that is, we think of it as being a point particle.

\hea Well, I think we need to eliminate the resistance of air. But rocks are quite heavy objects compared to their size, so the air resistance really does not matter. Another thing we should eliminate is gravity. But I don't know how we could do that. do you?

\heb Yes and no! You asked a question that will take us very far. Really. But let us not go there yet. If we cannot eliminate gravity, the stone makes an arc. But if the stone moves really fast and we consider only a very small time scale, then gravity does not really have time to have any effect and we may neglect it. Or, we may imagine soomehow compensating the gravitational force with some other force. Or maybe we could just remove gravity somehow.

\hea Okay, let us assume there is no gravity. The the stone would probably go just straight.

\heb Yes it would. According to the principle of relativity, the equation describing the motion of the stone, or any free particle, cannot involve the velocity directly. If we have neglected the effect of the environment, even gravity, then the law describing the stone cannot involve the location of the particle directly and must be independent of the direction where the stone is going. These notions are usually called the homogenity and isotropy of space.

\heb So, write on the blackboard the simplest law, consistent with these requirements, that you can ever imagine!

\heb Come on, be brave.

\hea Hmm.
\begin{eqna}
    \ddot{x}=0
\end{eqna}

\heb \emph{Voil\`a!} You just invented Newton's first law: free particles move straight, with unchanging velocity.

\hea There is also another equations consistent with our requirements, for example \(\dddot{x}=0\). But that would mean that the particle did somehow remember its acceleration. It would also mean that even acceleration were not absolute, only the change of acceleration, or jerk. But our experience tells us that acceleration is absolute. Newton's first law is in great agreement with experiments.

\hea How could it be that the simplest answer tend to be the correct one? I mean, I do really feel that `If I know nothing obout it, what else would I guess', but it still feels a bit, or not even just a bit, muddy.

\heb Say no more. In philosophy its called \emph{Occam's razor} and in some areas of physics \emph{the minimal substitution principle}. I think it is just an integral part of common sense. Impossible to prove or argue for, but also impossible to live without.

\heb It is somewhat similar issue with the issue of probability theory. Why do you expect that out of very many throws of dice about one sixth are sixes?

\hea I have no idea. Or I would say that because of summetry. But I cannot say why symmetry of the dice implies symmetry of the probabilities.

\heb Exactly. It just feel obvious. Even the rigorous mathematical theory of probabilities and measures and stochastics just lean on that asumption.

\hea But it is a very good assumption. What is the chances that it is wrong?

\hea Wait. What did I just say? Probabilities seem to be as enigmatic as time.

\heb Yeah they does. But if we want to focus on time, let's move on. Maybe we can come back to probabilities at some point?.

\hea Yeah, let's do that.
\chapter{Some physics}
\section{Force and Lagrangians}
\heb If you had to affect the motion of the stone somehow, what would you do?

\hea I could for example push it.

\heb Yeah. You could generate force with your muscles and direct it with your limbs. Let us denote the force you excert on the stone by \(F\). It is pretty obvious that force is a vector: it has a direction and a magnitude. If the stone is otherwise free, how does it react to the force?

\hea If Newton's first law says that the velocity of the free particle remains constant, maybe a force changes it.

\heb Yes, yes, yes! Here it comes again, Occam's razor or just plain common sense, and writes
\begin{eqna}
    F=\ddot{x}.
\end{eqna}
Is this good? At least it reduces to the first law if \(F=0\).

\hea Different stones, for example a small and a big one, react differently to a force.

\heb Right, so let us add a constant \(m\), called mass, which quantifies that and write
\begin{eqna}
    F=m\ddot{x}.
\end{eqna}
This equation is called Newton's second law. The greater the mass of the particle, the more resistant it is for the changes in its motion.

\hea That seems reasonable. But what about \(F\), is it clear that it is absolute in the sense that the acceleration is but the velocity is not?

\heb No. But it is also not perfectly clear how to attach an arrow describing the force that comes out of your muscles. At least it is not as easy as drawing an arrow between two fixed points in space.

\hea Here we need to decide how to define things. We need to guess what would be a reasonable definition, and here it is not perfectly clear. Let us assume that the force \emph{is} absolute in the sense in which the acceleration is, and let us use the second law as a definition of force.

\hea Could different definitions lead to different physical results?

\heb Yes they could. That is because the force is something concrete, something that you relly produce with your body. Newton's secobnd law really says that the mass is the only way in which bodies reactions to force differ.

\heb We could for example let the mass depend on the acceleration. And that dependence could be different for different particles. We just guess the Newton's second.

\hea I'm not perfectly comfortable with this reasoning.

\heb I understand you, but the situation could be so much worse! Think about some theories of psychology.

\hea Are you dissing psychology?

\heb No, I'm just pointing out how lucky we are. Our task is so much easier.

\heb And I can tell you that Newton's second law is pretty good a guess---it is valid at very large range of phenomena.
\subsection{Particle in a gravitational field}
\heb Let us take a simple example: a stone thrown straight up in the air. This is a nice example because we need only one number to describe the position of the stone, the height \(h\). Could you write down the equation of motion?

\hea Well, I guess the only relevant force in the problem is the gravitational force. It points straight down. Let me call it \(G\). The equation of motion thus reads
\begin{eqna}
    G=m\ddot{h}.
\end{eqna}
The force \(G\) is probably constant; I don't feel any difference in the gravitational force between top and a bottom of a hill.

\heb Yes, we can take it to be constant. We can choose to mass \(m\) to be the unit of mass, so we do not need to write it and we have just \(G=\ddot{h}\). This is a very simple differential equation which obviously is solved by some second degree polynomial. Try it!

\hea Okay, I take
\begin{eqna}
    h(t)=at^2+bt+c
\end{eqna}
and substitute it to the equation. I get \(G=2a\). So \(a\) must be equal to \(G/2\) and \(b\) and \(c\) could be anything. This fits to my intuition according to which we need to specify the initial velocity and location to know the motion.

\heb Perfect!
\subsection{Kinetic energy}
\heb In physics there is a concept called work, denoted by \(W\). What does work bring into your mind?

\hea Thinking. Pain, tiredness, but also satisfaction. But yeah, that is probably not relevant. I also recognize the old--fashionend `blood, sweat and tears' stylee work, for example lifting sldfjsdlf .

\heb That old--fashioned work is physical work. If you for example raahaa a sfss, the work done is proportional to the distance the fdfd moves and to the force needed for raahing. So, at least if the particle moves in the same direction than the force points, we can define \(\dd W=F\cdot\dd x=F_\coa\dd x^\coa\).

\hea If the particle moves perpendicularly to the force, then I don't really do work; I just maintain the force. For example if I stand besides a railroad and push with a scateboard a train that passes along, the wheels of the scateboard making sure I don't get disturbed bu the motion of the train, I don't really do nothing. I just push. So your definition seemse reasonable even if the force and motion are not perpendicular to each other.

\heb Yes. Let us adopt that definition in general.

\heb If you push some object with a constant force and it moves with a constant velocity, we have
\begin{eqna}
    \dot{W}=\der{}{t}(F\cdot\dd x)=F\cdot\der{x}{t}=F\cdot\dot{x},
\end{eqna}
so the greater the velocity, the greater \(\dot{W}\), or power, is needed. This is in accordance with everyday observations: the faster you raahata teh rock, the more you sweat.

\heb Now think of accelerating a particle of mass \(m\) from rest to the velocity \(\dot{x}=v\). The work that we need to do, denoted in this context by \(T\), is
\begin{eqna}
    T=\int_0^v\dd W=\int_o^v F\cdot\dd x.
\end{eqna}
Newton's second law tells that \(F=m\ddot{x}\), so
\begin{eqna}
    T=\int_0^v m\der{\dot{x}}{t}\cdot\dd x=m\int_0^v\dd\dot{x}\cdot\der{x}{t}=m\int_0^v\dot{x}\cdot\dd\dot{x}.
\end{eqna}
Leibniz rule tells us that
\begin{eqna}
    \dd(\dot{x}^2)=\dd(\dot{x}\cdot\dot{x})=\dd\dot{x}\cdot\dot{x}+\dot{x}\cdot\dd x=2\dot{x}\cdot\dd\dot{x}
\end{eqna}
so we have
\begin{eqna}
    T=\puoli m\int_0^v\dd(\dot{x}^2)=\puoli m v^2.
\end{eqna}
This is called the kinetic energy of the particle. We can also write it as
\begin{eqna}
    T=\puoli mg_{\coa\cob}v^\coa v^\cob.
\end{eqna}
For the particle in the gravitational field this is just \(\puoli\dot{h}^2\) since we chose the unit of mass so that for that particle it is one.

\heb Obviously, if we have many particles the total kinetic energy is just the sum of individual kinetic energies, since we must accelerate each particle independently of the others. Labeling particles with \(i\), we have
\begin{eqna}
    T=\puoli\sum_i m_ig_{\coa\cob}v_i^{\coa}v_i^{\cob}
\end{eqna}
Now, we can take advantage of the tensor formalism we developed. The above sum is clearly linear in each of the velocities in it. If we have for example three particles, the sum is a linear function of six velocity vectors, that is, a tensor with six lower indices. We can therefore write
\begin{eqna}
    T=\puoli m_{\coa\cob\coc\cod\coe\cof}v_1^{\coa}v_1^{\cob}v_2^{\coc}v_2^{\cod}v_3^{\coe}v_3^{\cof}
\end{eqna}
where \(m\) is a tensor, mathematically the direct sum of all \(m_i g\)'s?, which takes care of calculating the kinetic energies of all the particles.

\heb And we can go even further! A you can see, we probably want to contract the velocities of all particles or none of them. Therefore denote three balls with just one and write
\begin{eqna}
    T=\puoli m_{\coa\cob}\dot{q}^\coa\dot{q}^\cob.
\end{eqna}
Here \(\dot{q}\) denotes collectively the velocities of all particles.

\hea Damn this formalism is beautiful.
\subsection{Configuration space}
\hea It looks like we have only one particle now. Funny.

\heb We can think that we really have only one particle, the `system particle', instead of the original \(N\) particles. The sysem particle moves in \(3N\)--dimensional so--called configuration space. Position in configuration space is often denoted by \(q\) and the velocity naturally by \(\dot{q}\).

\hea Can we write Newton's second law for the system particle?

\heb Surely we can. For one particle it can be written as
\begin{eqna}
    F=m\ddot{x}\orr F_\coa=mg_{\coa\cob}\ddot{x}^\cob.
\end{eqna}
For the system particle we should obviously write
\begin{eqna}
    F_\coa=m_{\coa\cob}\ddot{q}^\cob,
\end{eqna}
or
\begin{eqna}
    F=\dot{p}\wheree p_\coa=m_{\coa\cob}\dot{q}^\coa,
\end{eqna}
since the tensor \(m\) is constant, at least if the masses of the particles are constants.

\heb If we have only one particle, we have
\begin{eqna}
    p_\coa=mg_{\coa\cob}\dot{x}^\cob\orr p=m\dot{x}
\end{eqna}
and Newton's second law reads
\begin{eqna}
    F=\dot{p}=m\ddot{x}
\end{eqna}
as it should be.

\heb The quantity \(p\) is called momentum. It in some sense quantifies the amount of motion in the system: the greater the mass and the greater the velocity, the greater the momentum is. It is proportional to the velocity, and therefore is not an absolute vector like the acceleration is.

\heb In terms of momentum the second law can be read as `Force equals the rate of change of momentum,' that is, force changes the amount of motion.

\hea Sounds very reasonable.
\subsection{Potential energy}
\heb We are not the onlyones who are able to produce force and do work. Mutual interactions of particles can often be thought of as arising from forces they excert on each other.

\heb Almost any large body can be thought of consisting of a large number of point particles excerting forces on each other. The forces keep the body in shape, but also may give rise for example for vibrations.

\hea So a complicated body can be represented as a point paricle in a large configuration space!

\heb Yes. The mutual forces of the particles appear an external forces in the configuration space.

\heb The force in the configuration space depend on the state of motion of the system. I could for example depend on \(q\), \(\dot{q}\) and \(\ddot{q}\). However, here we have to restrict to a narrow class of forces, but don't worry, actually the forces we will restrict on are the most interesting forces.

\hea What forces we abandon?

\heb Right now let us abandon forces that depend on the changing of \(q\), that is, we consider forces that only depend on \(q\).

\heb For such a force \(F\) we can calculate how much work must be done to change the configuration from \(q=A\) to \(q=B\). The work \(W\) is
\begin{eqna}
    W(A\rightarrow B)=\int_{A}^{B}F\cdot\dd q.
\end{eqna}
This may in general depend on the path taken, but if it does not, we can form a function \(V\) of the configuration \(q\) by setting
\begin{eqna}
    V(q)=W(A\rightarrow q).
\end{eqna}
Now we have%pitaa tsekata merkkejá
\begin{eqna}
    F\cdot\dd q=-\dd V=-\partial V\cdot\dd q,
\end{eqna}
where \(\partial V\) is the gradient of a \(V\), a vector that points points in the direction in which \(V\) grows most fast, its length of course depending on this rate.

\heb Newton's second law can therefore be written as
\begin{eqna}
    \dot{p}+\partial V=0\orr m_{\coa\cob}\ddot{q}^\cob+\partial_\coa V=0.
\end{eqna}
Looks simlpe, doesn't it!

\heb Now the kinetic energy can be written in a similar from.ldfcdlf
\begin{eqna}
    p_\coa=m_{\coa\cob}\dot{q}^\coa=\puoli\dot{\partial}m_{\coa\cob}\dot{q}^\coa\dot{q}^\cob=\dot{\partial}_\coa T.
\end{eqna}
Newton's second law becomes
\begin{eqna}
    \der{}{t}\dot{\partial}T+\partial V=0
\end{eqna}
or, after defining a new function \(L(q,\dot{q}=T-V\), the Lagranngian of the system,
\begin{eqna}
    \der{}{t}\dot{\partial}L-\partial L=0.
\end{eqna}
The equation of motion in this form is called the Lagrange's equation of motion.

\hea What can you say about the potential energy of our particle in a gravitational field?

\heb The force is constant, so it is almost trivial that it is the derivative of some constant \(\lambda\) times \(h\). The Lagrangian of the stone is therefore
\begin{eqna}
    L=\puoli\dot{h^2}-\lambda\,h.
\end{eqna}
\section{Action}
\heb The Lagrangian is actually very cool and also useful function. Think about for example the vertically moving stone. If we negllect gravity, the Lagrangian is ridiculously simple, just
\begin{eqna}
    L=\puoli\dot{h}^2.
\end{eqna}
The equation of motion is very simple, isn't it?

\hea Well yes, there are no forces so Newton's second law holds and \(\ddot{h}=0\). The velocity of the stone is constant.

\heb Along the motion the Lagrangian has some value at every moment, so we can condsider its time integral. It is called action of the motion and is denoted by \(S\). Clearly \(S\) is just the time the motion takes times the average value of the Lagrangian.

\heb Let's say we know something specific about the motion. We know the stone was at two particular moments \(t_a\) and \(t_b\) at at the heights \(h_a\) and \(h_2\).

\hea Ok. So from this information we immediately can calculate the velocity of the stone since we know the stone moves with a constant velocity. And also the average value of the Lagrangian.

\heb Now let us imagine that the stone did not have some constant velocity, but the motion still satisfies our boundary conditions. Such a motion is of course fictious, but we're equipped with an imagination. How does the action \(S\) change?

\hea Hmm. I don't know, it probably depends on our imagination on a very complicated way.

\heb But there is something that definitely happens. Think about the absolute value of the velocity.

\hea Aha! The stone must do some zig zag and on average it must thus have a greater absolute value of the velocity. That means that the value of the action gets bigger.

\heb So the realistic motion has the smalles \(S\) compared to the fictious motions.

\hea if we now turn the gravity on, the Lagrangian gets a new term \(-\lambda h\). For simplicity now assume that \(h_a=h_b\). What can you say about the motion?

\heb It must first go up and then down, all in a smooth fashion. We solved this and the answer was a parabola.

\heb Now the avarage of the absolute value of the velocity is not anymore the smallest of all imaginable motions. Now the real motion does not anymore have the smallest average of the absolute value of the velocity of all the imaginable motions. But the potential energy term is negative and compensates.

\hea Do you mean that by touring high in the potential the stone again manages to make the value of the action the lowest?

\heb At least it seems so! If the stone visited at even higher altitudes that it actually does, it would probably make the average speed too high.

\hea I see, and a parabola is probably the best compromise of having a fairly low average speed and a firly high potential at the same time.

\heb We can put this so--called principle off least action on a firm ground.
\subsection{Variation of the action}
\heb Let us assume that we have a system which has a Lagrangian \(L(q,\dot{q}\) and the motion of the systems satisfies Lagrange's equation of motion. The action is
\begin{eqna}
    S=\int L(q,\dot{q})\,\dd t.
\end{eqna}
Now let us change the motion \(q(t)\) by a small amount, that is, we displace the stone at every moment a small amount \(\delta q(t)\).

\heb The value of the Lagrangian changes a small amount \(\delta L\). The change is of course a function of time, but I do not bother to write the time dependece always explicitly. The change in the action is clearly
\begin{eqna}
    \delta S=\int\delta L\,\dd t.
\end{eqna}
Basic differential calculus tells us that
\begin{eqna}
    \delta L=\partial L\cdot\delta q+\dot{\partial}L\cdot\delta\dot{q}.
\end{eqna}
All these are of course functions of time.

\hea What is \(\delta\dot{q}\)? I guess we can specify only \(\delta q\).

\heb It is of course determined by \(\delta q\). See
\begin{eqna}
    \delta\dot{q}=\der{(q+\delta q)}{t}-\der{\delta q}{t}=\der{\delta q}{t}.
\end{eqna}

\hea Oh yes.

\heb We defined \(p=\dot{\partial}L\), so now we have
\begin{eqnb}
    \delta L&=&\partial L\cdot\delta q+p\cdot\der{}{t}\delta q\\
            &=&\partial L\cdot\delta q+\der{}{t}(p\cdot\delta q)-\dot{p}\cdot\delta q\\
            &=&(\partial L-\dot{p})\cdot\delta q+\der{}{t}(p\cdot\delta q).
\end{eqnb}
This holds for any motion \(q(t)\), were it realistic or not, and any small variation \(\delta q(t)\). But we assumed that \(q(t)\) is a realistic motion andd satisfies the equation of motion, so the first term vanishes. The second term is a time derivative, so it is trivial to integrate it over time. We get
\begin{eqna}
    \delta S=[p\cdot\delta q]_A^B.
\end{eqna}
But when we talked about the stone, we considered only motions with the same boundary conditions, right! If we compare the realistic motion only to motions that are almost realisic, and begin and end at exactly same configuration, then \(\delta q(t_a)=\delta q(t_b)=0\) and therefore also \(\delta S=0\)!

\subsection{Noether's theorem}
\heb If we can change the motion of a physical system in such a way that changed motion is also a realistic, the system is said to possess a symmetry. We can for example think of the motion of a thrown rock. We can move the throw to another city, or perhaps a hundred meters up, and get another possible motion of the rock.

\hea So you ar trying to say that the rock flies in the same way independently of the city and altitude where it has been thrown? I know that. You managed to choose pretty complicated words for expressing such a simple fact.

\heb Please excuse me. The flying rock is saide to possess a translation invariance. But the flight of the rock is an arc. If you turn it on its side, we get a flight which turns lef or right but not down---the rock clearly do not possess roational invariance around an axe parallel to ground.

\heb The translational invariance of the rock is a continuous symmetry, meaning that if we wish, we can do only a very small translation.

\heb A realistic motion, for example an arc of the rock, has two significant, though equivalent, properties: it satisfies the equation of motion and the value of the action is stationary for it. So, what quantifies our symmetry transformed motion?

\hea Obviously the same two properties.

\heb Exactly. Say we perform a small symmetry transformation \(\delta q\) to the motion. The value of the action remains stationary with respecto variations which vanish at the endpoints if the change in the action depends only on the endpoints, that is,
\begin{eqna}
    \delta S=f(A)-f(B)
\end{eqna}
for some \(f(q,\dot{q})\). This means that it must hold that \(\delta L=\dot{f}\). On the other hand, for any small change in a realistic motion we got
\begin{eqna}
    \delta L=\der{}{t}(p\cdot\delta q).
\end{eqna}
In totality we have
\begin{eqna}
    \dot{f}=\der{}{t}(p\cdot\delta q),
\end{eqna}
in other words the quantity
\begin{eqna}
    p\cdot\delta q-f
\end{eqna}






% ALL symmetries of the flying stone


% Constrained motion / Lagrange's multipliers

% The range of applicability of the Lagrangian method


%In one dimension \(\partial_x f(x)\) is a dual vector, because \(\partial_x f\dd x=\dd f\) is a number.

\begin{comment}
\chapter{Coordinating things}
\section{Basis}
\section{Isometries}
\section{Groups}

\chapter{Vibrations}
\section{The simple armonic oscillator}
\section{Vibrations of several degrees of freedom}
\newpage


\heb Let us consider some observers like you and me. Let's assume that the observers do not move and they all use same units of time and distance. How can these observers differ from each other?

\hea Well, I think everybody is different. Even if two persons were cloned from the same DNA, they have necessarily faced different things in life and become different kind of persons.

\heb Very clever. I wouldn't have known that! But you know that I meant something simpler.

\hea The observers are located at different places.

\heb And their noses may point in different directions. It is interesting to think of how the perspectives of these different observers are related to each other.

\heb We human beings have two eyes and therefore a stereo vision. We can locate things accurately in space by just watching. Say there is a couple of stones around and every observer see them. The question is: how are their points of view, I mean literally the image that gets projected on their verkkokalvo, relate to each other?

\hea Sounds like a complicated question.

\heb Well, the actual transformation of the view may be complicated, but we can say something about them. Let us denote the image on the verkkokalvos of some observer by \(\ket{\Psi}\). This notation is from quantum mechanics; I chose it because it looks so cool. Here it only means the view and nothing quantum mechanical.

\heb If we for example translate the viewer, that is change her location, the view changes. Let us denote the translation by \(\vec{T}\) and write
\begin{eqna}
    \ket{\Psi}\rightarrow\vec{T}\ket{\Psi}.
\end{eqna}

\hea Sounds trivial.

\heb I know. Now we can make another translation \(\primed{\vec{T}}\). In totality we have
\begin{eqna}
    \ket{\Psi}\rightarrow\vec{T}\ket{\Psi}\rightarrow\primed{\vec{T}}\vec{T}\ket{\Psi}.
\end{eqna}
Clearly both translations together form another translation which we may think of as a product \(\primed{\vec{T}}\vec{T}\). We can also think of doing nothing as a special kind of translation and denote it by \(1\).

\heb This is to say that mathematically the translations form a group. It is clear that the order of translations does not matter; we have \(\primed{\vec{T}}\vec{T}=\vec{T}\primed{\vec{T}}\) for any two translations. Translations are said to commute.

\hea I've heard of groups. If we translate by \(T\) we can always translate back, which I guess we could denote by \(T^{-1}\). It holds that \((\primed{T}T)^{-1}=T^{-1}\primed{T}^{-1}\) because
\begin{eqna}
    T^{-1}\primed{T}^{-1}\primed{T}T=T^{-1}1T=1.
\end{eqna}
This is clear because translating and then translating back is doing nothing. I feel stupid when I say that out loud.

\heb Good. Things get more interesting when we consider rotations. Rotations also clearly form a group which in three dimensions is known as \(SO(3)\).

\heb The group of rotations is more interesting because it is not commutative. Look: If I take this chair and do two \(\circc/4\) rotations around vertical and horizontal axes, I get different results depending on the order.

\hea I see, but what is \(\circc\)?

\heb It is the circle constant \(\circc\doteq2\pi\). \(\pi\) is used only for historical reasons; \(\circc\) is much more natural as it represens the whole circle. Check \url{tauday.com} for exhaustive discussion. If you don't like this, just go to the sourcecode of our lives and uncomment one line. The source can be found at \url{konstakurki.co/time}.

\hea Cool.






































\chapter{Curvature}
\heb If we take two parallel staight lines in space and extend them, do they remain parallel?

\hea Of course.

\heb How do you know that?

\hea Well, hmm, it seems obvious, doesn't it? I can't see how else could it be. Every straight lines that I've seen too be parallel at some point have always been parallel also everywhere else.

\heb OK, so you understand that it is an empirical observation. And then you of course also understand that it is not necessarily perfectly true, since we have really seen \emph{very} long straight parallel lines. Our picture of our space may be a bit too ideal

\hea Well, in principle yes, but this sounds like nitpicking to me. You said that usually the most simple guess is true, so I propose that parallel lines continue to be parallell since I think it is simpler than `parallel lines may stop being parallel'.

\heb That's surely what I did say. But what really is simple does not necessarily loook like that at first sight. This way or that, I really do have a point.

\heb Look at this karttapallo. Take two straight routes from the equator to the north pole. In the beginning they are parallel but they cross at the north pole and are definitely not parallel there.

\hea Ha ha. This really is stupid. The routes are of course really not straight; a straight route would go away from the surface of Earth to space. Earth is a ball in our three--dimensional space.

\heb Are you absolutely sure about that? Maybe we assume it is so because we have noticed that parallel lines do no continue as parallel and in reality our space is more peculiar that we assume.

\hea I think we can safely assume that light goes truly straight, and we know that a ship sailing away will eventually sink under the horizon. Earth really is a ball.

\heb Well, that is solid I must admit. Earth is a ball. But let me ask you another question. If we take an arrow, parallel transport it, that is transport it without changing the direction it points to, and parallel transport it back, does it end up pointing in the same direction to which it pointed in the beginning?

\hea I could of course ask how you know that the direction does not change, but I guess a gyroscope can be used to handle that issue. If we transport a gyroscope, hell yeah it will come back pointing at the original direction.

\heb Again, look at this model of Earth. Take an arrow and start from the equator, and go up to the north pole. Then go straigh right as long as the arrow is back on the equator. Then go back to the starting point along the equator. The arrow has turned one quarter of a full turn \(\circc\)!

\hea That's an interesting remark. But if you had really parallel transported the arrow, it would have pointed straight away from Earth at the north pole. You are again fooled by the fact that Earth is a ball.

\heb Yes I am. But now I tell you something that's gonna make you shit in your pants.

\heb Gyroscopes have actually been carried on satellites around Earth, and the results show that their directions \emph{do} change.

\hea What?

\heb Yes they do. If you don't believe, check the study of this and that. It can be found at arXiv under the number \arxivreference{1234.5678}.

\hea Don't worry, I believe you. But how is that possible?

\heb Well, that is something that will make us wonder a lot. But we may think that our space is actually intrinsically curved like the surface of Earth is.

\hea Is our space actually floating in some higher dimensional space?

\heb Maybe yes, maybe not. But it is actually not relevant for us, like the fact that Earth is embedded in a three--dimensional space is not relevant for a cartoghaper who just wants to draw useful two--dimensional pictures of land, seas and infrastructure built by human beings.

\heb We therefore are better to get used to curved spaces, or manifolds as they are often called in mathematics.






























\section{Riemannian manifolds}
\heb In mathematics a Riemannian manifold is a smooth space in which every curve has some length associated with it. Smooth means that the space is essentially flat if we look only small portions of it.

\heb For example the surface of Earth is a two--dimensional Riemannian manifold. It is smooth; if we take a small piece of this karttapallo it definitely is flat like a piece of paper. We can measure thee length of a curve for example by walking along it and counting the steps.

\heb But for example a cone is not smooth because the kärki is sharp no matter how close we look and thus is is not a Riemannian mmanifold.

\hea Seems like an intuitive concept. I guess our space is a three--dimensional Riemannian manifold?

\heb The truth is actually even more facinating, but right now we regard it as such. In any case manifolds will be vital for us if we want to understand time.
\subsection{Parallel transport}
\subsection{Vectors}
\heb Think of two points on a manifold so close to each other that the manifold is essentially flat on the scale of their separation. We can draw an arrow from the first to second. Let me denote the arrow, which is naturally called the separation of the two points, by \(\dd x\).

\heb Now if there are other two points near the first ones, with a separation \(\dd y\), we can naturally add them by parallel transporting the other to start from the end of the other. We get another vector. Because the manifold is essentially flat in the region containing the vectors there is no ambiguity in carrying out the parallel transport.

\hea You're addoing vectors!

\heb Yes I am!

\hea We could also multiply \(\dd x\) by a number \(a\) producing a separation of two points along the same line as the originals but \(a\) times further apart of each other. And so on, yes, I know about vectors.

\heb The separation \(\dd x\) is naturally a vecor, but we can also have other vectors. For example if we have a particle moving in a manifold, we can consider its location at two different times separated by a small time interval \(\dd t\). If the particle does not move unreasonably fast, the locations at \(t\) and \(t+\dd t\) are close together. Dividing this separation \(\dd x\) by \(\dd t\) we get another arrow
\begin{eqna}
    \der{x}{t}\doteq u,
\end{eqna}
the velocity, which is also clearly a vector.

\hea But now \(\dd t\) is very short and so \(u\) might stretch so far that the space cannot anymore be considered as flat?

\heb Well, for general vectors we imagine the small essentially flat region to extend to infinity as perfectly flat. Imagine gluing a flat piece of paper on the karttapallo. In mathematics this flat piece is called a tangent space. If you are interested in mathematical details, consider for example the excellent book \emph{Geometry, Topology and Physics} by Mikio Nakahara.

\hea Ok, I'll check it. Now, if I'm correct, each point have its own tangent space. We must think of a vector to actually be located to somewhere in the manifold, and only vectors located at the same point, and belonging to the same tangent space, can be added, ultimately because it is not clear how to move vectors over long distances.

\heb Exactly, exactly. Now think of the inner product \(A\cdot B\) of two vectors \(A\) and \(B\).

\hea It measures how much \(A\) and \(B\) overlap. It is symmetric, \(A\cdot B=B\cdot A\), and linear in its both arguments. If \(A\) and \(B\) are perpendicular to each other, we have \(A\cdot B=0\).

\heb Yes. Now it is interesting to take the inner product of \(\dd x\) with itself. We can also do the same for \(a\dd x\), which is \(a\) times longer separation. We get
\begin{eqna}
    (a\dd x)\cdot(a\dd x)=a^2\,\dd x\cdot\dd x
\end{eqna}
because the inner product is linear in both of its arguments.

\hea I'm getting this! An inner product can be used to measure the distance of closely separated points.

\heb Yes. We can equip tangent spaces with any inner product we want. In Riemannian geometry the inner product is chosen so that
\begin{eqna}
    \dd s^2=\dd x\cdot\dd x
\end{eqna}
where \(\dd s\) is the distance of two points separated by \(\dd x\).

\heb The inner product is a bilinear map which takes two vectors and produces a number. It is useful to consider linear maps from any number of vectors to real number. Such objects are called tensors.

\heb The Riemannian inner product is usually denoted by \(g\). It is called the metric tensor, because it is used to measure distance. In mathematics literature one usually writes \(A\cdot B\doteq g(A,B)\), but this kind of notation gets complicated if we have tensors with more `inputs' for vectors. I like to write it as
\begin{eqna}
    A\cdot B\doteq g_{\coa\cob}A^\coa B^\cob.
\end{eqna}
Because the inner product is symmetric in its both arguments, we have \(g_{\coa\cob}=g_{\cob\coa}\).

\hea What are those colored balls?

\heb They are wireless connectors that connect the vectors to the inputs of the multilinear map \(g\). The position of the ball determines the input used and color determines to which it is connected.

\hea We can also think of a tensor \(T\) which take only one vector \(V\) and produces a number. That would be written as \(T_\coa V^\coa\). Tensors like \(T\) are called dual vectors.

\heb We can also think of the vector \(V\) as a linear map which takes \(T\) as its input. It really doesn't matter. That is why \(T\) is called a dual vector.

\heb We can also think of first taking an outer product \(T_\coa V^\cob\) and then connecting \(_\coa\) to \(^\cob\). It is easy to verify that the outer product is actually a bilinear map which which takes one vector and one dual vector as its inputs.

\hea I see. We can take tensors with any number of balls upstairs and downstairs, for example \(\tensay{F}{\coa}{\cob}\) and \(\tensya{K}{\coc}{\cod}\). Then we can form the outer product \(\tensay{F}{\coa}{\cob}\tensya{K}{\coc}{\cod}\), which takes inputs exactly as you would expect from the balls.

\heb Yes.

\hea Then we can connect balls, for example \(\tensay{F}{\coa}{\cob}\tensya{K}{\coc}{\coa}\).

\heb Connecting the balls is often called contracting. Mathematically it means taking a trace. An upper ball must always be contracted with a lower one and vice versa.

\hea And the contracted indices are not anymore inputs! A contraction takes one input and one output off a tensor.

\heb finally, when there are no iputs and out puts left, we have just a number. How simple is that?

\hea Very simple and elegant.

\heb In this notation we have
\begin{eqna}
    \dd s^2=g_{\coa\cob}\dd x^\coa\dd x^\cob.
\end{eqna}
\subsection{Length of a curve}
Now, think of a curve \(x(t)\) on a manifold. We can calculate its length \(s\) by thinking it consisting of very many small segments. We have
\begin{eqna}
    s=\int\dd s=\int\sqrt{g_{\coa\cob}\dd x^\coa\dd x^\cob}.
\end{eqna}
\section{Differentiation}
\hea Could we differentiate things?

\heb If we have a scalar function \(\phi(x)\)on a manifold, that is, we attach a number to each point in the manifold, it is easy to compare neighboring values: just take the difference \(\dd\phi\). If we have a curve \(x(t)\), we can easily calculate the derivative of \(\phi\) with respect to \(t\); it is just \(\dd\phi/\dd t\).

\heb But differentiating a vector field is more difficult because we have to compare vectors belonging to different tangent spaces. How could we do it?

\hea Well, the only natural way of doing it that comes to my mind is of parallel transporting one of the vectors near the other and then doing the comparison.
\section{Differentiation}
\heb If we have tensors defined on a curve or perhaps everywhere on the manifold, it would be nice to be able to differentiate them. But there is a problem: we have to compare neighboring vectors that belong to different tangent spaces. How could we do it?

\hea Well, if we compare vectors the only thing that comes to my mind is of parallel transporting one of the vectors near the other and then doing the comparison. But I don't know what to do to other kinds of tensors.

\heb Damn you're clever. The difference of a vector field \(A(x)\) gotten that way is usually denoted by \(DA\). The capital d is used to emphasize that it is not a simple subtraction of two numbers. Now, if we have a curve \(x(t)\), we can defferentiate \(A\) along it by forming the simple fraction
\begin{eqna}
    \frac{DA}{\dd t}\doteq\cder{A}{t}.
\end{eqna}
It is also a vector. We usually write \(D\) instead of \(\dd\) also before \(t\) because mixing two different d's looks stupid.

\hea But what about other tensors?

\heb Oh yes. Actually they can also be naturally parallel transported, meaning that if the original tensor gives some number out of some arbitrary vectors, the parallel transported tensor gives the same number out of the same vectors also parallel transported to the same location.

\hea Like always, the covariant derivative obeys the Leibniz rule.

\end{comment}






\end{document}
